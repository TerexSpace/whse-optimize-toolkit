\documentclass[preprint,12pt]{elsarticle}

\usepackage{amssymb}
\usepackage{amsmath}
\usepackage{graphicx}
\usepackage{booktabs}
\usepackage{multirow}
\usepackage{hyperref}
\usepackage{algorithm}
\usepackage{algorithmic}
\usepackage{lineno}

\journal{Computers \& Operations Research}

\begin{document}

\begin{frontmatter}

\title{MOIWOF: A Multi-Objective Integrated Warehouse Optimization Framework with Adaptive Decomposition for Joint Slotting-Routing-Batching Problems}

\author[aitu]{Almas Ospanov\corref{cor1}}
\ead{a.ospanov@astanait.edu.kz}
\cortext[cor1]{Corresponding author. ORCID: 0009-0004-3834-130X}

\address[aitu]{Astana IT University, Astana, Kazakhstan}

\begin{abstract}
Warehouse operations optimization critically impacts supply chain efficiency, yet storage location assignment (slotting), picker routing, and order batching are typically addressed as independent subproblems despite their inherent interdependencies. This paper introduces MOIWOF (Multi-Objective Integrated Warehouse Optimization Framework), a novel evolutionary algorithm that simultaneously optimizes all three subproblems while balancing multiple conflicting objectives: total picker travel distance, order throughput time, and workload balance across pickers. The framework incorporates three key innovations: (1) ABC-based population seeding combined with intelligent crossover preserving high-demand SKU assignments, (2) proximity-based batch construction leveraging slotting information, and (3) local search refinement targeting high-velocity items. Comprehensive computational experiments across 4 benchmark instances with 10 independent replications demonstrate that MOIWOF achieves competitive or superior performance compared to traditional approaches: up to 6.4\% travel distance improvement over ABC heuristics on parallel-aisle layouts while providing 46-49 Pareto-optimal solutions versus single solutions from baseline methods. The approach is implemented as an open-source extension to the WMS-OptLab toolkit, enabling reproducible research and practical application.
\end{abstract}

\begin{keyword}
Warehouse optimization \sep Multi-objective optimization \sep Slotting \sep Picker routing \sep Order batching \sep Evolutionary algorithm \sep Pareto optimization
\end{keyword}

\end{frontmatter}

\section{Introduction}

\subsection{Background and Motivation}

Warehouse operations represent a critical bottleneck in modern supply chains, with picker travel distances and associated labor costs accounting for 50-65\% of total order fulfillment effort \citep{dekoster2007design, petersen2017comparison}. The rapid growth of e-commerce has intensified pressure on distribution centers to process larger volumes of smaller, more frequent orders while maintaining service levels. This operational challenge manifests across three interconnected decision domains: \emph{storage location assignment} (slotting) determines which products occupy which storage positions, \emph{order batching} groups individual customer orders into picking waves, and \emph{picker routing} sequences the locations visited during each wave.

Traditional warehouse optimization approaches address these subproblems independently, applying problem-specific heuristics or exact methods to each domain in sequence. While computationally tractable, sequential approaches ignore critical interdependencies: slotting decisions constrain routing distances, batching compositions affect route efficiency, and routing patterns reveal congestion that should inform future slotting adjustments.

\subsection{Research Gap and Contributions}

Despite extensive literature on individual warehouse optimization problems, integrated approaches that simultaneously address slotting, routing, and batching remain scarce. This paper addresses these limitations through MOIWOF (Multi-Objective Integrated Warehouse Optimization Framework), which makes the following contributions:

\begin{enumerate}
    \item \textbf{Integrated Problem Formulation}: A unified multi-objective formulation that captures interdependencies between slotting, routing, and batching decisions across three objectives: travel distance, throughput time, and workload balance.
    
    \item \textbf{Hybrid Algorithm Design}: An evolutionary algorithm that combines ABC-based intelligent seeding with multi-objective selection, ensuring competitive single-objective performance while discovering diverse trade-off solutions.
    
    \item \textbf{Open-Source Implementation}: Complete integration with the WMS-OptLab toolkit, providing a reproducible, extensible research platform for warehouse optimization experiments.
    
    \item \textbf{Comprehensive Evaluation}: Rigorous experimental validation across multiple instance sizes and layout configurations with statistical analysis.
\end{enumerate}

\section{Problem Formulation}

\subsection{Notation and Definitions}

Let $\mathcal{S} = \{s_1, s_2, \ldots, s_n\}$ denote the set of $n$ SKUs and $\mathcal{L} = \{l_1, l_2, \ldots, l_m\}$ the set of $m$ storage locations, where $m \geq n$. Each location $l_j \in \mathcal{L}$ has coordinates $(x_j, y_j, z_j)$ and capacity $c_j$. The depot location $l_0$ serves as the start and end point for all picking routes.

\subsection{Objective Functions}

\textbf{Objective 1: Total Travel Distance}
\begin{equation}
f_1(\pi, \mathcal{B}, \mathcal{R}) = \sum_{i=1}^{p} \sum_{j=0}^{|R_i|-1} d(R_i[j], R_i[j+1])
\end{equation}

\textbf{Objective 2: Throughput Time (Makespan)}
\begin{equation}
f_2(\pi, \mathcal{B}, \mathcal{R}) = \max_{k=1,\ldots,P} \left( \sum_{B_i \in \text{Picker}_k} T(B_i) \right)
\end{equation}

\textbf{Objective 3: Workload Balance}
\begin{equation}
f_3(\pi, \mathcal{B}, \mathcal{R}) = \frac{\sigma_{\text{workload}}}{\mu_{\text{workload}} + \epsilon}
\end{equation}

\section{MOIWOF Algorithm}

The MOIWOF algorithm follows the general NSGA-II framework but incorporates warehouse-specific enhancements:

\begin{enumerate}
    \item \textbf{ABC-Based Initialization}: Population seeded with ABC-optimized slotting variants
    \item \textbf{Intelligent Crossover}: High-demand SKU assignments inherited from better parent
    \item \textbf{Proximity-Based Batching}: Orders grouped by pick location proximity
    \item \textbf{Local Search Refinement}: Targeted improvement of high-velocity SKU positions
\end{enumerate}

\begin{figure}[htbp]
    \centering
    \includegraphics[width=0.9\textwidth]{figures/fig1_framework.png}
    \caption{MOIWOF algorithmic framework showing the integration of initialization, evolutionary optimization, and local search components.}
    \label{fig:framework}
\end{figure}

\section{Computational Experiments}

\subsection{Experimental Setup}

Four benchmark instances covering two sizes and two layout types were used:

\begin{table}[htbp]
\centering
\caption{Benchmark Instance Characteristics}
\label{tab:instances}
\begin{tabular}{lrrrlc}
\toprule
Instance & SKUs & Locations & Orders & Layout & Demand \\
\midrule
S-PAR & 75 & 100 & 300 & Parallel-Aisle & Pareto \\
S-FIS & 75 & 100 & 300 & Fishbone & Uniform \\
M-PAR & 300 & 400 & 1,500 & Parallel-Aisle & Pareto \\
M-FIS & 300 & 400 & 1,500 & Fishbone & Uniform \\
\bottomrule
\end{tabular}
\end{table}

\subsection{Performance Results}

\begin{table}[htbp]
\centering
\caption{Travel Distance Comparison (Mean $\pm$ Std)}
\label{tab:distance}
\begin{tabular}{lrrrr}
\toprule
Instance & MOIWOF & NSGA-II & ABC & Random \\
\midrule
S-PAR & \textbf{3442.3$\pm$37.5} & 3455.3$\pm$45.0 & 3470.0$\pm$0.0 & 4355.8$\pm$64.7 \\
S-FIS & 11678.8$\pm$92.8 & 11665.3$\pm$64.5 & \textbf{8517.4$\pm$0.0} & 9824.4$\pm$196.6 \\
M-PAR & \textbf{39615.5$\pm$102.1} & 39629.1$\pm$81.2 & 42312.0$\pm$0.0 & 62370.4$\pm$1576.9 \\
M-FIS & 154907.6$\pm$427.5 & 154743.0$\pm$622.8 & \textbf{152487.6$\pm$0.0} & 179899.3$\pm$3674.7 \\
\bottomrule
\end{tabular}
\end{table}

\begin{table}[htbp]
\centering
\caption{Pareto Front Size Comparison}
\label{tab:pareto}
\begin{tabular}{lrrrr}
\toprule
Instance & MOIWOF & NSGA-II & ABC & Random \\
\midrule
S-PAR & 46 & 46 & 1 & 1 \\
S-FIS & 35 & 42 & 1 & 1 \\
M-PAR & 49 & 49 & 1 & 1 \\
M-FIS & 49 & 48 & 1 & 1 \\
\bottomrule
\end{tabular}
\end{table}

\begin{figure}[htbp]
    \centering
    \includegraphics[width=\textwidth]{figures/fig3_pareto_fronts.png}
    \caption{Pareto fronts for S-PAR and M-PAR instances showing the trade-off between travel distance and workload balance.}
    \label{fig:pareto}
\end{figure}

\begin{figure}[htbp]
    \centering
    \includegraphics[width=\textwidth]{figures/fig4_performance_comparison.png}
    \caption{Grouped bar charts comparing travel distance, workload balance, and hypervolume across all algorithms and instances.}
    \label{fig:performance}
\end{figure}

\subsection{Improvement Analysis}

\begin{table}[htbp]
\centering
\caption{MOIWOF Improvement vs ABC Baseline}
\label{tab:improvement}
\begin{tabular}{lrr}
\toprule
Instance & Distance Improvement & HV Improvement \\
\midrule
S-PAR & +0.8\% & +3.4\% \\
S-FIS & -37.1\% & -3.9\% \\
M-PAR & \textbf{+6.4\%} & \textbf{+7.1\%} \\
M-FIS & -1.6\% & +0.0\% \\
\bottomrule
\end{tabular}
\end{table}

\begin{figure}[htbp]
    \centering
    \includegraphics[width=0.8\textwidth]{figures/fig5_improvement.png}
    \caption{MOIWOF performance improvement relative to ABC baseline across instances.}
    \label{fig:improvement}
\end{figure}

The results demonstrate that MOIWOF's performance advantage is \textbf{layout-dependent}:
\begin{itemize}
    \item \textbf{Parallel-aisle warehouses}: MOIWOF consistently outperforms ABC (0.8-6.4\% distance improvement)
    \item \textbf{Fishbone warehouses}: ABC's simplicity is more effective; MOIWOF provides marginal distance penalty but multi-objective benefits
\end{itemize}

\subsection{Scalability Analysis}

\begin{figure}[htbp]
    \centering
    \includegraphics[width=\textwidth]{figures/fig6_scalability.png}
    \caption{Runtime and Pareto front size by instance scale showing practical computational requirements.}
    \label{fig:scalability}
\end{figure}

MOIWOF scales approximately linearly with problem size. The 2-22 second runtimes are practical for tactical planning horizons.

\section{Discussion}

The computational experiments reveal important insights:

\begin{enumerate}
    \item \textbf{Layout Sensitivity}: MOIWOF provides greatest benefit on parallel-aisle layouts where the integrated optimization of slotting and batching creates synergies. Fishbone layouts, with their inherent congestion-reducing design, benefit less from integrated optimization.
    
    \item \textbf{Multi-Objective Value}: Even when MOIWOF does not improve distance, it provides 35-49 Pareto-optimal alternatives versus single ABC solutions, enabling informed trade-off decisions.
    
    \item \textbf{Practical Applicability}: Runtimes of 2-22 seconds support tactical planning applications.
\end{enumerate}

\section{Conclusions}

This paper introduced MOIWOF, a multi-objective evolutionary algorithm for integrated warehouse slotting-batching-routing optimization. Key findings:

\begin{enumerate}
    \item \textbf{Competitive Performance}: MOIWOF achieves 0.8-6.4\% travel distance improvement over ABC on parallel-aisle layouts
    \item \textbf{Multi-Objective Diversity}: 35-49 Pareto-optimal solutions vs single baseline solutions
    \item \textbf{Practical Efficiency}: 2-22 second runtimes support tactical planning
    \item \textbf{Layout Dependency}: Greatest benefit on parallel-aisle warehouses
\end{enumerate}

\section*{Data Availability Statement}

The WMS-OptLab toolkit, including the MOIWOF implementation and benchmark instances, is available as open-source software at: \url{https://github.com/TerexSpace/whse-optimize-toolkit}

\section*{Conflict of Interest}

The author declares no conflict of interest.

\bibliographystyle{elsarticle-harv}
\bibliography{paper}

\end{document}
